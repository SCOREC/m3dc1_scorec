\documentclass[11pt]{article}  % include bezier curves
\renewcommand\baselinestretch{1.0}           % single space

\oddsidemargin -10 true pt      % Left margin on odd-numbered pages.
\evensidemargin 10 true pt      % Left margin on even-numbered pages.
\marginparwidth 1.5 true in    % Width of marginal notes.
\oddsidemargin  0 true in       % Note that \oddsidemargin=\evensidemargin
\evensidemargin 0 true in
\topmargin -0.5 true in        % Nominal distance from top of page to top of
\textheight 9.0 true in         % Height of text (including footnotes and figures)
\textwidth 6.5 true in        % Width of text line.
\parindent=0pt                  % Do not indent paragraphs
\parskip=0.15 true in
%\usepackage{color}              % Need the color package
\usepackage{hyperref}
\usepackage[pdftex]{color, graphicx}
%\usepackage{graphicx} %for epsfig
\usepackage{setspace}
\usepackage{graphics}
\usepackage{subfig}
\usepackage{epsfig}
\usepackage{latexsym}
\usepackage{mathrsfs}
\usepackage{eufrak}
\usepackage{amsmath}
\usepackage{graphicx}
\usepackage{longtable}
\usepackage{afterpage}
\usepackage{pdfpages}
\usepackage{amsmath,url}
\title{Finite Element Procedure in M3D-C1}
\author{Fan Zhang, Mark S. Shephard , Seegyoung Seol} 
\begin{document} 
\maketitle
%\tableofcontents

\section{Introduction} \label{sec:intro}
M3D-C1 applies C$^1$ finite elements to solve the $4^{th}$ order PDEs derived from the magnetohydrodynamics (MHD) equations ~\cite{jardin2004triangular, jardin2012multiple} . Two finite element procedures that treat the essential boundary condition differently are discussed in the document.
  
The primary linear systems of equations formed by the current  procedure take the form as 
\begin{equation}
S \mathbf{u}^{n+1} = D \mathbf{u}^n +\mathbf{O}, \label{eqn:sys}
\end{equation}
where $S$ and $D$ are matrices, $\mathbf{u}^{n+1}$ and $\mathbf{u}^{n}$ are solution vectors at step $n+1$ and $n$ respectively, and $\mathbf{O}$ is a vector of the same size as $\mathbf{u} $. The physics quantity that $\mathbf{u}$ corresponds to include the velocity field, the magnetic field, the pressure and the density. 

The solution vector  $\mathbf{u}^{n+1}$ contains both the DOFs unknown and the DOFs prescribed by the essential boundary condition. The global matrix $D$ is explicitly formed to perform the matrix-vector multiplication. An alternative approach is discussed in the document. The essential boundary condition is  taken care of during the assembly of the global equation. The solution vector in the linear system of equations contains only DOFs unknown. The global matrix $D$ does not need to be formed.

Section \ref{sec:symbs} defines symbols and corresponding variables used in the document. Section \ref{sec:weakform} defines the weak form. Section \ref{sec:current} discusses the current procedure of the finite element analysis in M3D-C1. Section \ref{sec:alt} discusses the alternative approach that follows the lecture note. Section \ref{sec:equal} shows the equivalence of the two approaches.
 \section{Definition of Symbols} \label{sec:symbs}
 The symbols and corresponding variables in the pseudo code used in the document are listed below.
 \begin{description}
 \item $\Gamma_g$:  boundary of the domain that applies essential boundary condition;
 \item $\mathcal{\delta} =\{u|\, u\in H^{2},u|_{\Gamma_g}=g\}$:  trail solution space that satisfies the priori essential boundary condition; 
 \item $\mathcal{V}=\{\mu|\, \mu\in H^{2},u|_{\Gamma_g}=0\}$:  test function space;
 \item $\mathcal{\delta}^h$: discretized trail solution space;
 \item $\mathcal{V}^h$: discretized test function space;
 \item $a(\cdot, \cdot)$: bilinear form respect to the two ``$\cdot$'' in the expression.;
 \item $(\cdot, f)$: linear form with respect to the ``$\cdot$'' in the expression;
 \item $u^{n}$: solution  at time step n (known);
 \item UN = \{$\mathbf{u}^{n}$: solution  vector in terms of DOFs at time step n\};
 \item ${u}^{n+1}$: solution  at time step n+1 (unknown);
 \item UNP = \{$\mathbf{u}^{n+1}$: solution vector in terms of DOFs at time step n+1\};
 \item NNP = \{$N_{np}$: number of nodes mesh\};
 \item NEL = \{$N_{el}$: number of elements mesh\};
 \item NEN = \{$N_{en}$: number of nodes per element (3 for triangle, 6 for wedge)\};
 \item NDN = \{$N_{dn}$: number of DOFs per mesh node\};
 \item NEE = \{$n_{ee}$: number of DOFs per element ($n_{ee}=N_{en}N_{dn}$)\};
 \item KE = \{$k^e$: element stiffness matrix ($n_{ee} \times n_{ee}$)\}; 
 \item FE = \{$\mathbf{f}^e$: element force vector  ($n_{ee}$)\};
 \end{description}
 The symbols used in Section \ref{sec:current} are listed below.
 \begin{description}
 \item NDOFG = \{$N_{dofg}$: number of possible DOFs\};
 \item S = \{$S$: the left-hand-side matrix in Equation \ref{eqn:sys} ($n_{dofg} \times n_{dofg}$)\};
 \item D = \{$D$: the right-hand-side matrix  in Equation \ref{eqn:sys} ($n_{dofg} \times n_{dofg}$)\};
 \item O = \{$\mathbf{O}$: the force vector in terms of DOFs in Equation \ref{eqn:sys} (($n_{dofg}$))\}.
 \end{description}
 
 The symbols used in Section \ref{sec:alt} are listed below.
 \begin{description}
 \item $N_a$: shape function associated with the mesh entity not on $\Gamma_g$;
 \item $N_b$: shape function associated with the mesh entity on $\Gamma_g$;
 \item NDOF = \{$N_{dof}$: number of unknown DOFs\};
 \item NDOG = \{$N_{dog}$: number of prescribed non-zero DOFs\};
 \item K = \{$K$: global stiffness matrix ($n_{dof} \times n_{dof}$) \};
 \item F = \{$\mathbf{F}$: global force vector ($n_{dof}$)\};
 \item d = \{$\mathbf{d}$: vector of unknown DOFs\}.
 \end{description}
\section{Weak Form} \label{sec:weakform}
The weak form of the PDEs  \cite{hughes2012finite} solved by M3D-C1 is summarized below. Only the essential  boundary condition is applied in M3D-C1. The  weak formulation can be written as the follows.
 
\textit{ Find $u^{n+1}\in \mathcal{\delta}^h$, such that for all $w \in \mathcal{V}^h$,}
\begin{equation}
a(w, u^{n+1}) = \tilde{a}(w, u^n) + (w,f). \label{eqn:weakform}
\end{equation}
We assume the spacial dimension as one ($N_{sd}=1$) such that we drop the vector notation of $u$ for simplicity. 

 \section{Current Finite Element Procedure} \label{sec:current}
Expend $u^{n+1}$, $u^n$, and $w^h$ in terms of shape functions.
\begin{eqnarray}
u^{n+1}&=&\sum_{i=1}^{N_{dofg}} N_{i} u_{i}^{n+1}\\
u^n &=&\sum_{i=1}^{N_{dofg}} N_{i} u_{i}^{n} \\
w^{h}&=&\sum_{i=1}^{N_{dof}} N_{ai} c_{i},
\end{eqnarray}
where $c_i\in \mathcal{R}$ is arbitrary.

The finite element analysis procedure is illustrated as below. The sub-routines called by the procedure are defined in the following sub-sections.
\begin{samepage}
\begin{verbatim}
procedure FEA-MHD
    var NDN = {number DOF per node}
    var UN ={solution vector at n step}
    var UNP={solution vector at n+1 step}
    var KE={element stiffness from a(.,.)}
    var TKE={element stiffness from tilde_a(.,.)}
    var FE={element force vector from (w,f)}
    var S, D ={global matrices in Equation 1}
    var O ={global vector in Equation 1}
    
    call ORDER_DOF (NDN) 
         {label global equation number}
    
    initialize UN at step 0
    
    for I_STEP = 1 to T
    {from time t1 to total time T}
        for ELEMENT = 1 to NEL {loop over elements}
            call ELE_STIFF(ELEMENT, KE, TKE, FE) {calculate element stiffness}
            call ASSEMBLE(KE, TKE, FE, S, D, O) {assemble}
        end for
        APPLY_BC(S, O) {apply boundary condition} 
        O=D*UN+O {matrix-vector multiplication + element force}
        solve UNP=S{-1} O  {obtain the solution}
        UN=UNP
    end for
end procedure
\end{verbatim}
\end{samepage}
 \subsection{DOF ordering}

Given the number of DOFs per mesh node, NDN ($N_{dn}$), the DOF ordering procedure assigns an equation number at each possible nodal DOF.
\begin{samepage}
\begin{verbatim}
procedure ORDER_DOF(NDN)
    var NDN={number of DOFs per node}
    var P=1
    for each mesh node
        for p=1 to NDN
        assign P as the equation number
        P=P+1 
        end for
    end for
end procedure
\end{verbatim}
\end{samepage}
\subsection{Element stiffness procedure}
The following element contributions are calculated.
 \begin{itemize}
 \item $K^e$  from $a(\cdot, \cdot)$ (variable KE);
 \item $\tilde{K}^e$  from  $\tilde{a}(\cdot, \cdot)$  (variable TKE);
 \item $\mathbf{f}^e$ (FE) from  $(\cdot, f)$ (variable FE).
 \end{itemize}
\begin{samepage}
\begin{verbatim}
procedure ELE_STIFF(ELEMENT, KE, TKE, FE)
    var NEN={number nodes per element}
    var NODES[NEN]
    var row=1, column
    get the nodes of the element in NODES
    for i=1 to NEN
        for p=1 to NDN
             row = row+1
             if p is constrained, continue
             {if constrained DOF, skip the loop}
             calculate FE(row)
             column=1
             for j=1 to NEN
                 for q=1 to NDN
                     column = column +1
                     calculate KE(row, column)
                     calculate TKE (row, column)
                end for
            end for
        end for
    end for
end procedure
\end{verbatim}
\end{samepage}
 \subsection{Element Assembly}
 The element assembly procedure assembles the element contributions to the global matrix $S$, $D$ and the global vector $\mathbf{O}$. The capital letter (P,Q) are the global equation numbers  of the local DOF indices (p,q) from the DOF ordering. 
\begin{samepage}
\begin{verbatim}
procedure ASSEMBLE(KE, TKE, FE, S, D, O)
for p=1 to NEE
    if p is constrained, continue
    {if constrained DOF, skip the loop}
    add FE(p) to O(P)
    for q=1 to NEE
        add KE(p,q) to S(P,Q)
        add TKE(p,q) to D(P,Q)
    end for
end for
\end{verbatim}
\end{samepage}
\subsection{Apply boundary conditions}
The boundary condition is applied by putting a ``one'' at the diagonal of the row  in the matrix $S$ that the corresponding DOF is constrained. The corresponding row of the vector $\mathbf{O}$ is set to the constrained value. 
\begin{samepage}
\begin{verbatim}
procedure APPLY_BC(S, O)
var NDOFG={number of possible DOFs}
for P=1 to NDOFG
    if DOF P constrained 
        S(P,P)=1
        O(P)=UN(P)
    end if
end for
\end{verbatim}
\end{samepage}
 \subsection{Linear solve in M3D-C1}
 The linear solve assumes the variation from the solution at time 0 is small.  The element contributions are approximated by the solution at time 0 and they are only calculated at the first step.
 \begin{samepage}
\begin{verbatim}
 procedure FEA-MHD-LINEAR
    initialize UN at step 0
  
    {the first step}
    for I_EL=1 to NEL {loop over elements}
        call ELE_STIFF(ELEMENT, KE, TKE, FE) {calculate element stiffness}
        call ASSEMBLE(KE, TKE, FE, S, D, O) {assemble}
    end for
    APPLY_BC(S, O) {apply boundary condition}
    O=D*UN+O {matrix-vector multiplication + element force}
    solve S UNP = O {obtain the solution at first step}
    UN=UNP
    
    for I_STEP = 2 to T
    {from time t2 to total time T}
        APPLY_BC(O) {apply boundary condition only to O} 
        O=D*UN+O
        solve S UNP = O {obtain the solution}
        UN=UNP
    end for
end procedure
\end{verbatim}
\end{samepage}
 \section{Alternative Finite Element Procedure} \label{sec:alt}
Start with the Equation \ref{eqn:weakform}.  $u^{n+1}$ can be written as
\begin{equation}
u^{n+1}=v^h+g^h,
\end{equation}
where $v^h\in \mathcal{V}^h$ and $g^h\in \mathcal{\delta}^h$. It yields
\begin{equation}
a(w, v^{h}) = \tilde{a}(w, u^n) - a(w, g^h)+ (w,f).
\end{equation}

Expend $v^h$, $g^h$, $u^n$ and $w$ in terms of shape functions.
\begin{eqnarray}
v^h&=&\sum_{i=1}^{N_{dof}} N_{ai} d_{i}^{n+1}\\
g^h&=&\sum_{i=1}^{N_{dog}} N_{bi} g_i \\
u^n &=&\sum_{i=1}^{N_{dof}} N_{ai} d_{i}^{n}+\sum_{i=1}^{N_{dog}} N_{b} g_i \\
w^{h}&=&\sum_{i=1}^{N_{dof}} N_{ai} c_{i},
\end{eqnarray}
where $c_i\in \mathcal{R}$ is arbitrary.

Using the property of the bilinear and linear forms, the weak form is written in the term of the linear system of equations as the follows.

 \textit{Find $\mathbf{d}^{n+1}$, such that
 \begin{equation}
 \sum_{j=1}^{N_{dof}}  a(N_{ai}, N_{aj} d_{j}^{n+1}) = \sum_{j=1}^{N_{dof}}  \tilde{a}(N_{ai}, N_{aj} d_{j}^{n})  + \sum_{j=1}^{N_{dog}}  \tilde{a}(N_{ia}, N_{b} g_j) - \sum_{j=1}^{N_{dog}} a(N_{ai},  N_{bj} g_j )+ (N_{ai},f).
 \end{equation}
 for  $i=1,2,...N_{dof}$. }
 
 The assembly procedure is discussed. The input element contributions are
 \begin{itemize}
 \item $K^e$  from $a(\cdot, \cdot)$ (variable KE),
 \item $\tilde{K}^e$  from  $\tilde{a}(\cdot, \cdot)$  (variable TKE),
 \item $\mathbf{f}^e$ (FE) from  $(\cdot, f)$ (variable FE).
 \end{itemize}
Variable UN is the solution vector in the terms of DOFs at step n, $\mathbf{u}^n$. 
Notice the global prescribed displacement vector (denoted as G in the lecture note) is contained in UN.
 
 The aim is to assemble the contributions to the global stiffness matrix $K$ and the global force vector $\mathbf{f}$ (F). We use capital letter (P,Q) as the location in the global matrix or vector of the local DOF indices (p,q). 
 The procedure is illustrated as below.
\begin{samepage}
\begin{verbatim}
procedure ASSEMBLE(KE, TKE, FE, UN, KF)

for p=1 to NEE
    if row associated DOG or NULL, continue
    {skip the local rows of DOG and NULL}
    for q=1 to NEE
        add TKE(p,q)UN(Q) to F(P)
        if column associated DOF
            add KE(p,q) to K(P,Q)  
        end if
        if column associated DOG 
            add -KE(p,q)UN(Q) to F(P)
        end if
    end for
    add FE(p) to F(P)
end for
\end{verbatim}
\end{samepage}
\section{Equivalence of Two methods in Matrix Form} \label{sec:equal}
The linear system of equations formed in Section \ref{sec:current} can be written in the matrix form as
\begin{equation}
 \left[ \begin{array}{cc}
          K_{ff} & K_{fg}\\
                 & I
       \end{array}
  \right]
  \left\lbrace \begin{array}{c}
         d_{f}\\
         d_{g}
      \end{array}
  \right\rbrace
  =
  \left\lbrace \begin{array}{c}
         F\\
         G
      \end{array}
  \right\rbrace
\end{equation}

where $I$ is the identity matrix.
It is straightforward to see the linear system is equivalent to
\begin{equation}
 \left[ \begin{array}{c}
          K_{ff} 
       \end{array}
  \right]
  \left\lbrace \begin{array}{c}
         d_{f}
      \end{array}
  \right\rbrace
  =
  \left\lbrace \begin{array}{c}
         F
      \end{array}
  \right\rbrace - 
   \left[ \begin{array}{c}
          K_{fg} 
       \end{array}
  \right]
  \left\lbrace \begin{array}{c}
         G
      \end{array}
  \right\rbrace
\end{equation}

\bibliographystyle{ieeetr}
\bibliography{reference}
\end{document}